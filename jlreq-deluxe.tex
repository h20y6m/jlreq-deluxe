\documentclass[dvipdfmx,a4paper,jis2004]{jlreq}

\usepackage[T1]{fontenc}
\usepackage{lmodern}

\usepackage{jlreq-deluxe}
\usepackage[haranoaji]{pxchfon}
% 原ノ味フォント / Harano Aji Fonts (https://github.com/trueroad/HaranoAjiFonts)

\usepackage{bxghost}

\usepackage{shortvrb}
\MakeShortVerb{\|}

\newcommand{\pkg}[1]{\textsf{#1}}
\newcommand{\cls}[1]{\textsf{#1}}
\newcommand{\cmd}[1]{\eghostguarded{\texttt{\symbol{92}#1}}}
\newcommand{\meta}[1]{$\langle$\nobreak\hspace{0pt}#1\nobreak\hspace{0pt}$\rangle$}

\newcommand{\pTeX}{p\TeX}
\newcommand{\pLaTeX}{p\LaTeX}
\newcommand{\upTeX}{u\pTeX}
\newcommand{\upLaTeX}{u\pLaTeX}


\title{\pkg{jlreq-deluxe}パッケージ}
\author{h20y6m}
\date{2020/03/15}

\begin{document}

\maketitle

\section{概要}

八登崇之氏作成の\pkg{pxjodel}パッケージ\footnote{\texttt{https://www.ctan.org/pkg/pxjodel}}%
を利用し\pLaTeX および\upLaTeX 上の\cls{jlreq}クラスにおいて和文を多書体(多ウェイト)にする
機能を提供する。

\section{前提条件}

\begin{itemize}
\item \TeX フォーマット:\LaTeX 。
\item \TeX エンジン:\pTeX 、\upTeX 。
\item DVIウェア:和文TFMとVFをサポートするもの。
\item 前提パッケージ:
  \begin{itemize}
  \item \cls{jlreq}クラス
  \item \pkg{pxjodel}パッケージ
  \end{itemize}
\end{itemize}

\section{使用方法}

通常のパッケージと同様に\cmd{usepackage}で読み込む。

\begin{quote}
\begin{verbatim}
\usepackage[オプション]{jlreq-deluxe}
\end{verbatim}
\end{quote}

基本的に\cls{jlreq}クラスとともに使用することを想定しているが、
他のクラスでも使用することは出来る。

\section{オプション}

基本的に\pkg{otf}パッケージのと同じオプションが使用できるが、
以下のオプションは動作が異なる。

\begin{itemize}
\item |deluxe|\\
  既定で有効になる。
  無効にしたい場合は|deluxe=false|を指定する。
\item |expert|\\
  使用できない。
\item |burasage|\\
  使用できない。
\item |scale|\\
  \cls{jlreq}クラスを使用している場合は自動的に設定され指定は無視される。
\end{itemize}

また以下のオプションが使用できる。

\begin{itemize}
\item |hanging_punctuation|\\
  \cls{jlreq}クラスの|hanging_punctuation|指定時用のJFMを使用する。
  \cls{jlreq}クラスを使用している場合は自動的に設定され指定は無視される。
\item |zenkakunibu_nibu|\\
  \cls{jlreq}クラスの|open_bracket_pos=zenkakunibu_nibu|指定時用のJFMを使用する。
  \cls{jlreq}クラスを使用している場合は自動的に設定され指定は無視される。
\end{itemize}

\end{document}