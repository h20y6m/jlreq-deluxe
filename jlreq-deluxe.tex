\documentclass[dvipdfmx]{jlreq}

\usepackage[T1]{fontenc}
\usepackage{lmodern}

\usepackage[haranoaji,jis2004]{jlreq-deluxe}
% 原ノ味フォント / Harano Aji Fonts (https://github.com/trueroad/HaranoAjiFonts)

\usepackage{bxghost}

\usepackage{shortvrb}
\MakeShortVerb{\|}

\newcommand{\pkg}[1]{\textsf{#1}}
\newcommand{\cls}[1]{\textsf{#1}}
\newcommand{\cmd}[1]{\eghostguarded{\texttt{\symbol{92}#1}}}
\newcommand{\meta}[1]{$\langle$\nobreak\hspace{0pt}#1\nobreak\hspace{0pt}$\rangle$}

\newcommand{\pTeX}{p\TeX}
\newcommand{\pLaTeX}{p\LaTeX}
\newcommand{\upTeX}{u\pTeX}
\newcommand{\upLaTeX}{u\pLaTeX}


\title{\pkg{jlreq-deluxe} パッケージ}
\author{h20y6m}
\date{2019/12/01}

\begin{document}

\maketitle

\section{概要}

{\pLaTeX}および{\upLaTeX}上の\cls{jlreq}クラスにおいて和文を多書体(多ウェイト)にする機能を提供する。

\section{前提条件}

\begin{itemize}
\item {\TeX}フォーマット:{\LaTeX}。
\item {\TeX}エンジン:{\pTeX}、{\upTeX}。
\item DVIウェア:dvipdfmx(フォント指定機能)
\item 前提パッケージ:
  \begin{itemize}
  \item \cls{jlreq}クラス
  \item \pkg{otf}パッケージ(\cmd{UTF}及び\cmd{CID}用のフォント)
  \end{itemize}
\end{itemize}

\section{オプション}

オプションは次のものが用意されている。

\begin{itemize}
\item \textsf{プリセット指定オプション}(|ipaex|、|hiragino-pron| 等)\\
  名前に対応するプリセット指定を有効にする。
\item |jis2004|\\
  フォント指定機能でJIS2004字形を使用する。
\item |unicode|\\
  フォント指定でUnicode直接指定を使用する。({\pLaTeX}では使用できない。)
\end{itemize}

\section{機能}

\subsection{多書体(多ウェイト)}

明朝・ゴシックともに3ウェイトおよび丸ゴシックが使用可能になる。

\begin{itemize}
\item 明朝・細ウェイト(|\mcfamily\ltseries|)
\item 明朝・中ウェイト(|\mcfamily\mdseries|)
\item 明朝・太ウェイト(|\mcfamily\bfseries|)
\item ゴシック・中ウェイト(|\gtfamily\mdseries|)
\item ゴシック・太ウェイト(|\gtfamily\bfseries|)
\item ゴシック・極太ウェイト(|\gtfamily\ebseries|)
\item 丸ゴシック(|\mgfamily|)
\end{itemize}

% \subsection{\cmd{UTF}及び\cmd{CID}}

% 以下の命令が使用できる。

% \begin{itemize}
% \item |\UTF{|\meta{十六進}|}|:\meta{十六進}で指定されたUnicodeポイントの文字を出力する。
% \item |\CID{|\meta{十進}|}|:\meta{十進}で指定されたCIDの文字を出力する。\\
%   ※|unicode|オプション指定時は使用できない。
%\end{itemize}

\subsection{フォント指定}

以下のフォント指定命令が使用できる。

\begin{itemize}
\item |\setminchofont[|\meta{番号}|]{|\meta{フォントファイル名}|}|:\\
  明朝体{\footnotesize (|\mcfamily|)}の3ウェイトすべてのフォントを置き換える。\\
  TTC形式の場合の該当フォントの番号を\meta{番号}に指定する。
\item |\setgothicfont[|\meta{番号}|]{|\meta{フォントファイル名}|}|:\\
  ゴシック体{\footnotesize (|\gtfamily|)}の3ウェイトすべてのフォントを置き換える。\\
  TTC形式の場合の該当フォントの番号を\meta{番号}に指定する。
\item |\setlightminchofont[|\meta{番号}|]{|\meta{フォントファイル名}|}|:\\
  明朝体・細ウェイト{\footnotesize (|\mcfamily\ltseries|)}のフォントを置き換える。\\
  TTC形式の場合の該当フォントの番号を\meta{番号}に指定する。
\item |\setmediumminchofont[|\meta{番号}|]{|\meta{フォントファイル名}|}|:\\
  明朝体・中ウェイト{\footnotesize (|\mcfamily\mdseries|)}のフォントを置き換える。\\
  TTC形式の場合の該当フォントの番号を\meta{番号}に指定する。
\item |\setboldminchofont[|\meta{番号}|]{|\meta{フォントファイル名}|}|:\\
  明朝体・太ウェイト{\footnotesize (|\mcfamily\bfseries|)}のフォントを置き換える。\\
  TTC形式の場合の該当フォントの番号を\meta{番号}に指定する。
\item |\setmediumgothicfont[|\meta{番号}|]{|\meta{フォントファイル名}|}|:\\
  ゴシック体・中ウェイト{\footnotesize (|\gtfamily\mdseries|)}のフォントを置き換える。\\
  TTC形式の場合の該当フォントの番号を\meta{番号}に指定する。
\item |\setboldgothicfont[|\meta{番号}|]{|\meta{フォントファイル名}|}|:\\
  ゴシック体・太ウェイト{\footnotesize (|\gtfamily\bfseries|)}のフォントを置き換える。\\
  TTC形式の場合の該当フォントの番号を\meta{番号}に指定する。
\item |\setxboldgothicfont[|\meta{番号}|]{|\meta{フォントファイル名}|}|:\\
  ゴシック体・極太ウェイト{\footnotesize (|\gtfamily\ebseries|)}のフォントを置き換える。\\
  TTC形式の場合の該当フォントの番号を\meta{番号}に指定する。
\item |\setmarugothicfont[|\meta{番号}|]{|\meta{フォントファイル名}|}|:\\
  丸ゴシック体{\footnotesize (|\mgfamily|)}のフォントを置き換える。\\
  TTC形式の場合の該当フォントの番号を\meta{番号}に指定する。
\end{itemize}

\section{プリセット指定}

以下のプリセット指定が使用できる。

\begin{itemize}
\item |ms|:MSフォント。
\item |ipa|:IPAフォント。
\item |ipaex|:IPAexフォント。
\item |ms-hg|:MSフォント+HGフォント。
\item |ipa-hg|:IPAフォント+HGフォント。
\item |ipaex-hg|:IPAexフォント+HGフォント。
\item |moga-mobo|:Mogaフォント+Moboフォント。
\item |moga-mobo-ex|:MogaExフォント+MoboExフォント。
\item |moga-maruberi|:Mogaフォント+モトヤLマルベリ3等幅。
\item |ume|:梅フォント。
\item |kozuka-pro|:小塚フォント(Pro版)。
\item |kozuka-pr6|:小塚フォント(Pr6版)。
\item |kozuka-pr6n|:小塚フォント(Pr6n版)。
\item |hiragino-pro|:ヒラギノ基本6書体セット(Pro/Std版)+明朝W2。
\item |hiragino-pron|:ヒラギノ基本6書体セット(ProN/StdN版)+明朝W2。
\item |hiragino-elcapitan-pro|:ヒラギノフォント(Mac OS X El Capitan搭載;Pro/Std版)+明朝W2。
\item |hiragino-elcapitan-pron|:ヒラギノフォント(Mac OS X El Capitan搭載;ProN/StdN版)+明朝W2。
\item |morisawa-pro|:モリサワフォント基本7書体(Pro版)。
\item |morisawa-pr6n|:モリサワフォント基本7書体(Pr6N版)。
\item |yu-win|:游書体(Windows 8.1搭載版)。
\item |yu-win10|:游書体(Windows 10搭載版)。\\
  ※“”‘’の出力が不正になる。|unicode|オプションを指定することで回避できる。
\item |yu-osx|:游書体(Mac OS X搭載版)。
\item |sourcehan-otc|:Source Han Serif(源ノ明朝)+Source Han Sans(源ノ角ゴシック)、OTC版。\\
  ※|unicode|オプションが自動的に有効になる。{\pLaTeX}では使用できない。
\item |sourcehan|:Source Han Serif(源ノ明朝)+Source Han Sans(源ノ角ゴシック)、言語別OTF版。\\
  ※|unicode|オプションが自動的に有効になる。{\pLaTeX}では使用できない。
\item |sourcehan-jp|:Source Han Serif(源ノ明朝)+Source Han Sans(源ノ角ゴシック)、地域別サブセットOTF版。\\
  ※|unicode|オプションが自動的に有効になる。{\pLaTeX}では使用できない。
\item |noto-otc|:Noto Serif CJK JP+Noto Sans CJK JP、OTC版。\\
  ※|unicode|オプションが自動的に有効になる。{\pLaTeX}では使用できない。
\item |noto|:Noto Serif CJK JP+Noto Sans CJK JP、言語別OTF版。\\
  ※|unicode|オプションが自動的に有効になる。{\pLaTeX}では使用できない。
\item |noto-jp|:Noto Serif CJK JP+Noto Sans CJK JP、地域別サブセットOTF版。\\
  ※|unicode|オプションが自動的に有効になる。{\pLaTeX}では使用できない。
\item |haranoaji|:原ノ味明朝+原ノ味ゴシック。
\end{itemize}

\end{document}
